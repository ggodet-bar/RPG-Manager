\subsection{Scene 1 - FBI interrogation}

\begin{scene_summary}
Will Sonntag is brought to the local police station, where the FBI has just
taken control of operations for investigating on a series of murders. This is
when Mr. Sonntag gets to know the FBI PC, Caleb Anders. While the interrogation
goes on, a power failure triggers the fire alarm and the security evacuation.
At this time a short manifestation of a demon occurs. This will be a fatal
meeting for Harry Endring, Caleb Anders' teammate. The scene should end as the
PCs are fleeing the office.
\end{scene_summary}

\begin{scene_tag}
PC: Caleb Anders, Will Sonntag
NPC: Harry Endring
Location: FBI office
\end{scene_tag}

\begin{leading_idea}
{\bf Fast and Anxious}

This scene should contrast vividly with the previous one. For the sake of its
dynamism, GMs should avoid throwing too much dices and focus on a nervous
narration.

{\bf Antagonism}

Preparing the players for the scenario becomes essential here. Simplistic
\href{http://deadgentlemen.com/projects/the-gamers/the-gamers/}{\textit{d\'enouement}},
\textit{\`a la} \textit{``you look trustworthy, would you like to join us?''},
is of course strongly discouraged. On the contrary, the GM should encourage
Caleb Anders' PC to be on their guard concerning Will Sonntag.  We suggest
handing Caleb's PC either notes or extracts from murder reports from the past
murders, as well as the clues that highlight an occult connection between the)
murders and Will Sonntag.
\end{leading_idea}

\begin{sidenote}
{\bf What if the interrogation goes wrong?}
The examination aims to link the PCs through a conflict situation. However, as
any examination, the situation may develop into something unwanted ... Do not
let the situation get away and feel free to rush the first effect of
the ritual.
\end{sidenote}

\begin{acting_directions}
Action must allow PCs to bond. Roleplay occurring during the examination
is a perfect time to reach this goal. In order to keep things under control,
the GM has an essential role in pushing the FBI character in the right
direction, or even in having NPCs intervening to interrupt, question etc.
\end{acting_direction}

{\bf What is undesirable effect of the ritual?}
The lights turn off  few seconds. Emergency lighting turn on rapidly to
illuminate with a dull green light the interrogation room. While everybody's
wondering about what's happening, a deep gurgling resounds at the back of the
room. The looks come back on a green tinted reality, attracted by a layer of a
dark yellow sticky sauce pouring on the table and dripping on the overturned
chairs. Seated on his chair, Caleb Anders' teammate eyes only reflects a frosty
vacuum. A strange shapeless bloody thing dirtily tries to penetrate into the
poor guy's chest. Clearly, the tentative wasn't a great success. Like a
caprice, the thing blurts out a shrill crying and with force and perseveration,
it disappeared in the bloody gap which throne in middle of the rest of the
chest.

\begin{sidenote}
{\bf About the ritual collateral effect}
The effect mustn't drag in length. It must be shock, an improbable vision which
takes end immediately, leaving an strange impression of madness and unreality.
\end{sidenote}

Will Sonntag (PC) is woken up at dawn by a small team of FBI investigators,
lead by Caleb Anders, knocking on his door and yelling his name. Needless to
say that by the time Sonntag wakes up, this is already their last warning and
the door is about to be blown up. Of course, the FBI team is there with a
mandate from a federal judge.

Whatever comes out of this sequence, Will Sonntag should be brought to the
FBI's local office (or whatever office they've taken over). As he is taken out
of the police car, a homeless bystander (in fact, [REF NPC] this is the first
apparition of Baron Hobo) starts screaming and blathering about the end of the
world, the advent of demonic forces. He finally points at Sonntag and calls him
Half-Antichrist, Child of Paradox, Incubus of Apocalypse (GMs are encouraged to
elaborate on this occult streak). While Sonntag is pushed into the corridors of
the station, the screams of the homeless madman can still be heard. A police
officer should ask Anders at some point what should be done with the man.

The scene carries on with the interrogation. In parallel to the interrogation
scene, a ritual murder takes place in a warehouse north of the city. The
interrogation is being stopped by one of the unexpected effects of the ritual
failed.

\begin{location_description}
The questioning takes place in a room in the offices of the FBI. A simple
rectangular table occupies the center of the room. Two chairs either side of
the table lets one realize without too much difficulty the usage of this
place.
\end{location_description}
