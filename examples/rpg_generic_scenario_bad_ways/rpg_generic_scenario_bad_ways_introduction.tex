\section{Introduction}

Master idea
the first player leaves the psychiatric hospital and meets the second player
(FBI). The introduction ends at the beginning of the questioning of the first
player by the second.

<<<<<<< HEAD
Adventure starts as Will Sonntag is given back his wallet and flat keys, as he
is leaving the Meyerling Institute. His editor, Jennyfer Hennings, is waiting for him in an
old corvette in front of the asylum. Being a old wisened bird from the
publishing industry, Jennyfer Hennings is quite anxious to put Will back in the writing
saddle. Even though Sonntag's case seems desperate, she still dreams of
retiring with a final success on her catalogue.

\begin{acting_directions}
Jennyfer Hennings is not a bad person. Maybe a bit anxious. Maybe she's had too much failures
in her carreer not to have been hardened by the blows. Will Sonntag's return in
the public life is her last chance, so the damn writer'd better do right away
what he's good at! Make her babble about all sorts of trivial things, but 
mention in passing the atrocious murders that have been committed downtown.
\end{acting_direction}

Jennyfer Hennings finally drops Will in front of his building. The PC should fill a bit at
loss, standing in a quiet Brooklyn street on a quiet autumn day. If the PC
turns on the radio or the TV or talks to neighbours, you should mention (once
again as smalltalk) the murders that have been going on since the previous
week. Note that this information should be drowned into a sea of more or less
menacing rumors of comparable weirdness.
=======
\begin{leading_idea}
This introduction scene should set both the general atmosphere (film noir) and
discretely introduce the murders committed by the PC's twin sister (and
official villain). This effect should be achieved by implementing lots of
smalltalk between the PC and the NPCs, or even between NPCs (e.g., Will
Sonntag's neighbours, his landlord etc.). Most of Tarantino's dialogue scenes
are a great example of how to get the audience to know the characters by
watching them react to anecdotes, jokes and trivial debates (e.g., the tipping
argument in \emph{Reservoir Dogs}). Additionally, the details on Will Sonntag
provided by the GM by way of the dialogue with Marcy Rice should give some
hints to the FBI PC for the interrogation scene.
\end{leading_idea}

The adventure starts as Will Sonntag is given back his wallet and flat keys, as
he is leaving the Meyerling Institute. His editor from \emph{Black Lilies
Editions}, Marcy Rice, is waiting for him in an old Corvette in front of the
asylum. Being a old wisened bird from the publishing industry, Marcy Rice is
quite anxious to put Will back in the writing saddle. Even though Sonntag's
case seems desperate, she still dreams of retiring with a final success on her
catalogue.

\begin{acting_directions}
Marcy Rice is not a bad person. Maybe a bit anxious. Maybe she's had too much
failures in her carreer not to have been hardened by the blows. Will Sonntag's
return in the public life is her last chance, so the damn writer'd better do
right away what he's good at! Make her babble about all sorts of trivial
things (for instance, Sonntag's only ``serious'' contender at \emph{Black
Lilies Editions}, Sonny Lirrah, has just had a close-call overdose, and Marcy
is feeling like she's running ``a damn rehab center, \emph{Naked Lunch}
style''), but mention in passing the atrocious murders that have been committed
downtown.
\end{acting_direction}

Marcy Rice finally drops Will in front of his building. The PC should fill a
bit at loss, standing in a quiet Brooklyn street on a quiet autumn day. If the
PC turns on the radio or the TV or talks to neighbours, you should mention
(once again as smalltalk) the murders that have been going on since the
previous week. Note that this information should be drowned into a sea of more
or less menacing rumors of comparable weirdness. The GM should let this feeling
sink as long as needed, until the PC chooses to get to bed.
>>>>>>> 36d9115cf8a40b4248e435c4269e2a4014c5b753

\begin{sidenote}
{\bf What if the PC leaves home?}

If the PC chooses to take a walk, goes to a bar, the GM should nevertheless
find ways to translate the ominous mood of the scene (e.g. the bar's TV is
turned on some terrible news, clients look miserable, etc.).
\end{sidenote}

\begin{sidenote}
{\bf Some weird news items}

\begin{itemize}
\item
\end{itemize}
\end{sidenote}

