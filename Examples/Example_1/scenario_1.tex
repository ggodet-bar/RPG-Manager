\documentclass{/.../...//class_scenario_1}

\title{This is a test}
\date{\today}

\begin{document}

\maketitle

\begin{background}

Les Chansons de Tom Bombadil
	
Hol�! Viens gai dol! sonne un donguedillon!
Sonne un dong! Saute! fal lall le saule!
Tom Bom, gai Tom, Tom Bombadillon!

Hol�! Viens gai dol! derry dol! Ch�rie!
L�gers sont le vent du temps et l' �tourneau ail�.
L�-bas sous la colline, brillante au soleil,
L� est ma belle dame, fille de Dame Rivi�re,
Mince comme la baguette de saule, plus claire que l' onde.
Le vieux Tom Bombadil, porteur de lis d' eau,
Rentre de nouveau en sautillant. L' entends-tu chanter?
Hol�! Viens gai dol! derry dol! et gai-ho,
Baie d' or, baie d' or, gaie baie jaune, oh!
Pauvre viel Homme-Saule, retire tes racines!
Tom est press� � pr�sent. Le soir va suivre le jour.
Tom rentre, porteur de lis d' eau.
Hol�! viens derry dol! M' entends-tu chanter?

Trottez mes petits amis, le long du Tournesaules;
Tom va devant allumer les chandelles.
A l' Ouest se couche le soleil : bient�t vous irez � l' aveuglette.
Quand tomberont les ombres de la nuit, la porte s' ouvrira;
Par les carreaux de la fen�tre, la lumi�re scintillera, jaune.
Ne craignez pas d' aulnes noirs! Ne vous souciez pas des saules chenus!
Ne craignez ni racine ni branche! Tom va devant vous.
Hol�, maintenant! Gai dol! On vous attendra!

Hol�! Venez gai dol! Sautez mes braves!
Hobbits! Poneys, tous! On aime les r�unions.
Que le plaisir commence! Chantons en choeur!

Que les chants commencent! Chantons en choeur
Le soleil, les �toiles, la lune et la brume, la pluie et le temps nuageux,
La lumi�re sur la feuille qui bourgeonne, la ros�e sur la plume,
Le vent sur la colline d�couverte, les cloches sur la brande,
Les roseaux pr�s de l' �tang ombreux, les lis sur l' eau :
Le vieux tom Bombadil et la fille de la Rivi�re!

Le vieux Tom Bombadil est un gai luron;
Bleu vif est sa veste, et ses bottes sont jaunes.

Oh�! Tom Bombadil, Tom Bombadillon!
Par l' eau, la for�t et la colline, par le roseau et le saule,
Par le feu, le soleil et la lune, �coutez maintenant et entendez-nous!
Accourez, Tom Bombadil, car notre besoin est proche de nous


Oh�! voyons! Venez, voyons. Hol�! O� vaquez-vous?
En haut, en bas, pr�s ou loin, ici, l� ou l�-bas?
Ou�e-fine, Bon-nez, Queue-vive et Godichon,
Paturons-blancs, mon petit gars, et toi, mon vieux Gros-Balourd!

http://www.elbakin.net/tolkien/essais/poesie12.htm
\end{background}

\end{document}  






